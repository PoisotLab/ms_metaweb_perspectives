%!TEX TS-program = xelatex
\documentclass[11pt]{article}

\usepackage[english]{babel}

\usepackage{amsmath,amssymb,amsfonts}
\usepackage[utf8]{inputenc}
\usepackage[T1]{fontenc}
\usepackage{stix2}
\usepackage[scaled]{helvet}
\usepackage[scaled]{inconsolata}

\usepackage{lastpage}

\usepackage{setspace}

\usepackage{ccicons}

\usepackage[hang,flushmargin]{footmisc}

\usepackage{geometry}

\setlength{\parindent}{0pt}
\setlength{\parskip}{6pt plus 2pt minus 1pt}

\usepackage{fancyhdr}
\renewcommand{\headrulewidth}{0pt}\providecommand{\tightlist}{%
  \setlength{\itemsep}{0pt}\setlength{\parskip}{0pt}}

\makeatletter
\newcounter{tableno}
\newenvironment{tablenos:no-prefix-table-caption}{
  \caption@ifcompatibility{}{
    \let\oldthetable\thetable
    \let\oldtheHtable\theHtable
    \renewcommand{\thetable}{tableno:\thetableno}
    \renewcommand{\theHtable}{tableno:\thetableno}
    \stepcounter{tableno}
    \captionsetup{labelformat=empty}
  }
}{
  \caption@ifcompatibility{}{
    \captionsetup{labelformat=default}
    \let\thetable\oldthetable
    \let\theHtable\oldtheHtable
    \addtocounter{table}{-1}
  }
}
\makeatother

\usepackage{array}
\newcommand{\PreserveBackslash}[1]{\let\temp=\\#1\let\\=\temp}
\let\PBS=\PreserveBackslash

\usepackage[breaklinks=true]{hyperref}
\hypersetup{colorlinks,%
citecolor=blue,%
filecolor=blue,%
linkcolor=blue,%
urlcolor=blue}
\usepackage{url}

\usepackage{caption}
\setcounter{secnumdepth}{0}
\usepackage{cleveref}

\usepackage{graphicx}
\makeatletter
\def\maxwidth{\ifdim\Gin@nat@width>\linewidth\linewidth
\else\Gin@nat@width\fi}
\makeatother
\let\Oldincludegraphics\includegraphics
\renewcommand{\includegraphics}[1]{\Oldincludegraphics[width=\maxwidth]{#1}}

\usepackage{longtable}
\usepackage{booktabs}

\usepackage{color}
\usepackage{fancyvrb}
\newcommand{\VerbBar}{|}
\newcommand{\VERB}{\Verb[commandchars=\\\{\}]}
\DefineVerbatimEnvironment{Highlighting}{Verbatim}{commandchars=\\\{\}}
% Add ',fontsize=\small' for more characters per line
\usepackage{framed}
\definecolor{shadecolor}{RGB}{248,248,248}
\newenvironment{Shaded}{\begin{snugshade}}{\end{snugshade}}
\newcommand{\KeywordTok}[1]{\textcolor[rgb]{0.13,0.29,0.53}{\textbf{#1}}}
\newcommand{\DataTypeTok}[1]{\textcolor[rgb]{0.13,0.29,0.53}{#1}}
\newcommand{\DecValTok}[1]{\textcolor[rgb]{0.00,0.00,0.81}{#1}}
\newcommand{\BaseNTok}[1]{\textcolor[rgb]{0.00,0.00,0.81}{#1}}
\newcommand{\FloatTok}[1]{\textcolor[rgb]{0.00,0.00,0.81}{#1}}
\newcommand{\ConstantTok}[1]{\textcolor[rgb]{0.00,0.00,0.00}{#1}}
\newcommand{\CharTok}[1]{\textcolor[rgb]{0.31,0.60,0.02}{#1}}
\newcommand{\SpecialCharTok}[1]{\textcolor[rgb]{0.00,0.00,0.00}{#1}}
\newcommand{\StringTok}[1]{\textcolor[rgb]{0.31,0.60,0.02}{#1}}
\newcommand{\VerbatimStringTok}[1]{\textcolor[rgb]{0.31,0.60,0.02}{#1}}
\newcommand{\SpecialStringTok}[1]{\textcolor[rgb]{0.31,0.60,0.02}{#1}}
\newcommand{\ImportTok}[1]{#1}
\newcommand{\CommentTok}[1]{\textcolor[rgb]{0.56,0.35,0.01}{\textit{#1}}}
\newcommand{\DocumentationTok}[1]{\textcolor[rgb]{0.56,0.35,0.01}{\textbf{\textit{#1}}}}
\newcommand{\AnnotationTok}[1]{\textcolor[rgb]{0.56,0.35,0.01}{\textbf{\textit{#1}}}}
\newcommand{\CommentVarTok}[1]{\textcolor[rgb]{0.56,0.35,0.01}{\textbf{\textit{#1}}}}
\newcommand{\OtherTok}[1]{\textcolor[rgb]{0.56,0.35,0.01}{#1}}
\newcommand{\FunctionTok}[1]{\textcolor[rgb]{0.00,0.00,0.00}{#1}}
\newcommand{\VariableTok}[1]{\textcolor[rgb]{0.00,0.00,0.00}{#1}}
\newcommand{\ControlFlowTok}[1]{\textcolor[rgb]{0.13,0.29,0.53}{\textbf{#1}}}
\newcommand{\OperatorTok}[1]{\textcolor[rgb]{0.81,0.36,0.00}{\textbf{#1}}}
\newcommand{\BuiltInTok}[1]{#1}
\newcommand{\ExtensionTok}[1]{#1}
\newcommand{\PreprocessorTok}[1]{\textcolor[rgb]{0.56,0.35,0.01}{\textit{#1}}}
\newcommand{\AttributeTok}[1]{\textcolor[rgb]{0.77,0.63,0.00}{#1}}
\newcommand{\RegionMarkerTok}[1]{#1}
\newcommand{\InformationTok}[1]{\textcolor[rgb]{0.56,0.35,0.01}{\textbf{\textit{#1}}}}
\newcommand{\WarningTok}[1]{\textcolor[rgb]{0.56,0.35,0.01}{\textbf{\textit{#1}}}}
\newcommand{\AlertTok}[1]{\textcolor[rgb]{0.94,0.16,0.16}{#1}}
\newcommand{\ErrorTok}[1]{\textcolor[rgb]{0.64,0.00,0.00}{\textbf{#1}}}
\newcommand{\NormalTok}[1]{#1}

\newlength{\cslhangindent}
\setlength{\cslhangindent}{1.5em}
\newlength{\csllabelwidth}
\setlength{\csllabelwidth}{3em}
\newenvironment{CSLReferences}[3] % #1 hanging-ident, #2 entry spacing
 {% don't indent paragraphs
  \setlength{\parindent}{0pt}
  % turn on hanging indent if param 1 is 1
  \ifodd #1 \everypar{\setlength{\hangindent}{\cslhangindent}}\ignorespaces\fi
  % set entry spacing
  \ifnum #2 > 0
  \setlength{\parskip}{#2\baselineskip}
  \fi
 }%
 {}
\usepackage{calc} % for \widthof, \maxof
\newcommand{\CSLBlock}[1]{#1\hfill\break}
\newcommand{\CSLLeftMargin}[1]{\parbox[t]{\maxof{\widthof{#1}}{\csllabelwidth}}{#1}}
\newcommand{\CSLRightInline}[1]{\parbox[t]{\linewidth}{#1}}
\newcommand{\CSLIndent}[1]{\hspace{\cslhangindent}#1}\geometry{verbose,letterpaper,tmargin=2.2cm,bmargin=2.2cm,lmargin=2.2cm,rmargin=2.2cm}

\usepackage{lineno}
\usepackage[nolists,noheads]{endfloat}

\pagestyle{plain}

\tolerance=1
\emergencystretch=\maxdimen
\hyphenpenalty=10000
\hbadness=10000

\doublespacing

\fancypagestyle{normal}
{
  \fancyhf{}
  \fancyfoot[R]{\footnotesize\sffamily\thepage\ of \pageref*{LastPage}}
}
\begin{document}
\raggedright
\thispagestyle{empty}
{\Large\bfseries\sffamily Template to prepare preprints and manuscripts
using markdown and github actions}
\vskip 5em

%
\href{https://orcid.org/0000-0002-0735-5184}{Timothée\,Poisot}%
%
\,\textsuperscript{1,2,‡}\quad %
Peregrin\,Took%
%
\,\textsuperscript{3,4}\quad %
Merriadoc\,Brandybuck%
%
\,\textsuperscript{5,4,‡}

\textsuperscript{1}\,Université de
Montréal\quad \textsuperscript{2}\,Québec Centre for Biodiversity
Sciences\quad \textsuperscript{3}\,Inn of the Prancing
Pony\quad \textsuperscript{4}\,Fellowship of the
Ring\quad \textsuperscript{5}\,Green Dragon Inn

\textsuperscript{‡}\,These authors contributed equally to the work\\

\textbf{Correspondance to:}\\
Timothée Poisot --- \texttt{timothee.poisot@umontreal.ca}\\

\vfill
This work is released by its authors under a CC-BY 4.0 license\hfill\ccby\\
Last revision: \emph{\today}

\clearpage
\thispagestyle{empty}

\vfill


        {\bfseries Purpose:}\,This template provides a series of scripts
to render a markdown document into an interactive website and a series
of PDFs.\\%
        {\bfseries Motivation:}\,It makes collaborating on text with
GitHub easier, and means that we never need to think about the
output.\\%
        {\bfseries Internals:}\,GitHub actions and a series of python
scritpts. The markdown is handled with \texttt{pandoc}.\\%
    
\vfill

\clearpage
\linenumbers
\pagestyle{normal}

\hypertarget{intro}{%
\section{Intro}\label{intro}}

Having a general solution for inferring \emph{potential} interactions
(despite the unavailability of interaction data) could be the catalyst
for significant breakthroughs in our ability to start thinking about
species interaction networks over large spatial scales. In a recent
overview of the field of ecological network prediction, Strydom \emph{et
al.} (2021) identified two challenges of interest to the prediction of
interactions at large scales. First, there is a relative scarcity of
relevant data in most places globally -- paradoxically, this restricts
our ability to infer interactions to locations where inference is
perhaps the least required; second, accurate predictions often demand
accurate predictors, and the lack of methods that can leverage small
amount of data is a serious impediment to our predictive ability
globally.

Following the definition of Dunne (2006), a metaweb is a network
analogue to the regional species pool; specifically, it is an inventory
of all \emph{potential} interactions between species from a spatially
delimited area (and so captures the \(\gamma\) diversity of
interactions). The metaweb is, therefore, \emph{not} a prediction of the
food web at a specific locale within the spatial area it covers, and
will have a different structure (notably by having a larger connectance;
see \emph{e.g.} Wood \emph{et al.} 2015). These local food webs (which
captures the \(\alpha\) diversity of interactions) are a subset of the
metaweb's species and interactions, and have been called ``metaweb
realizations'' (Poisot \emph{et al.} 2015). Differences between local
food web and their metaweb are due to chance, species abundance and
co-occurrence, local environmental conditions, and local distribution of
functional traits, among others.

Because the metaweb represents the joint effect of functional,
phylogenetic, and macroecological processes (Morales-Castilla \emph{et
al.} 2015), it holds valuable ecological information. Specifically, it
is the ``upper bounds'' on what the composition of the local networks
can be (see \emph{e.g.} McLeod \emph{et al.} 2021). These local
networks, in turn, can be reconstructed given appropriate knowledge of
local species composition, providing information on structure of food
webs at finer spatial scales. This has been done for example for
tree-galler-parasitoid systems (Gravel \emph{et al.} 2018), fish trophic
interactions (Albouy \emph{et al.} 2019), tetrapod trophic interactions
(O'Connor \emph{et al.} 2020), and crop-pest networks (Grünig \emph{et
al.} 2020). Whereas the original metaweb definition, and indeed most
past uses of metawebs, was based on the presence/absence of
interactions, we focus on \emph{probabilistic} metawebs where
interactions are represented as the chance of success of a Bernoulli
trial (see \emph{e.g.} Poisot \emph{et al.} 2016); therefore, not only
does our method recommend interactions that may exist, it gives each
interaction a score, allowing us to properly weigh them.

Yet, owing to the inherent plasticity of interactions, there have been
documented instances of food webs undergoing rapid collapse/recovery
cycles over short periods of time (Pedersen \emph{et al.} 2017). The
embedding of a network, in a sense, embeds its macro-evolutionary
history, especially as RDPG captures ecological signal (Dalla Riva \&
Stouffer 2016); at this point, it is important to recall that a metaweb
is intended as a catalogue of all potential interactions, which should
then be filtered (Morales-Castilla \emph{et al.} 2015). In practice (and
in this instance) the reconstructed metaweb will predict interactions
that are plausible based on the species' evolutionary history, however
some interactions would/would not be realized due to human impact.

Dallas \emph{et al.} (2017) suggested that most links in ecological
networks may be cryptic, \emph{i.e.} uncommon or otherwise hard to
observe. This argument essentially echoes Jordano (2016): the sampling
of ecological interactions is difficult because it requires first the
joint observation of two species, and then the observation of their
interaction. In addition, it is generally expected that weak or rare
links would be more common in networks (Csermely 2004), compared to
strong, persistent links; this is notably the case in food chains,
wherein many weaker links are key to the stability of a system (Neutel
\emph{et al.} 2002). In the light of these observations, the results in
fig.~\ref{fig:inflation} are not particularly surprising: we expect to
see a surge in these low-probability interactions under a model that has
a good predictive accuracy. Because the predictions we generate are by
design probabilistic, then one can weigh these rare links appropriately.
In a sense, that most ecological interactions are elusive can call for a
slightly different approach to sampling: once the common interactions
are documented, the effort required in documenting each rare interaction
may increase exponentially. Recent proposals suggest that machine
learning algorithms, in these situations, can act as data generators
(Hoffmann \emph{et al.} 2019): in this perspective, high quality
observational data can be supplemented with synthetic data coming from
predictive models, which increases the volume of information available
for inference. Indeed, Strydom \emph{et al.} (2021) suggested that
knowing the metaweb may render the prediction of local networks easier,
because it fixes an ``upper bound'' on which interactions can exist;
indeed, with a probabilistic metaweb, we can consider that the metaweb
represents an aggregation of informative priors on the interactions.

As Herbert (1965) rightfully pointed out, ``{[}y{]}ou can't draw neat
lines around planet-wide problems''; in this regard, our approach (and
indeed, any inference of a metaweb at large scales) must contend with
several interesting and interwoven families of problems. The first is
the limit of the metaweb to embed and transfer. If the initial metaweb
is too narrow in scope, notably from a taxonomic point of view, the
chances of finding another area with enough related species to make a
reliable inference decreases; this would likely be indicated by large
confidence intervals during ancestral character estimation, but the lack
of well documented metawebs is currently preventing the development of
more concrete guidelines. The question of phylogenetic relatedness and
dispersal is notably true if the metaweb is assembled in an area with
mostly endemic species, and as with every predictive algorithm, there is
room for the application of our best ecological judgement. Conversely,
the metaweb should be reliably filled, which assumes that the \(S^2\)
interactions in a pool of \(S\) species have been examined, either
through literature surveys or expert elicitation. Supp. Mat. 1 provides
some guidance as to the type of sampling effort that should be
prioritized. While RDPG was able to maintain very high predictive power
when interactions were missing, the addition of false positive
interactions was immediately detected; this suggests that it may be
appropriate to err on the side of ``too many'' interactions when
constructing the initial metaweb to be transferred. The second series of
problems are related to determining which area should be used to infer
the new metaweb in, as this determines the species pool that must be
used.

In our application, we focused on the mammals of Canada. The upside of
this approach is that information at the country level is likely to be
required by policy makers and stakeholders for their biodiversity
assessment, as each country tends to set goals at the national level
(Buxton \emph{et al.} 2021) for which quantitative instruments are
designed (Turak \emph{et al.} 2017), with specific strategies often
enacted at smaller scales (Ray \emph{et al.} 2021). And yet, we do not
really have a satisfying answer to the question of ``where does a food
web stop?''; the current most satisfying solutions involve examining the
spatial consistency of network area relationships (see \emph{e.g.}
Galiana \emph{et al.} 2018, 2019, 2021; Fortin \emph{et al.} 2021),
which is of course impossible in the absence of enough information about
the network itself. This suggests that an \emph{a posteriori} refinement
of the results may be required, based on a downscaling of the metaweb.
The final family of problems relates less to the availability of data or
quantitative tools, and more to the praxis of spatial ecology. Operating
under the context of national divisions, in large parts of the world,
reflects nothing more than the legacy of settler colonialism. Indeed,
the use of ecological data is not an apolitical act (Nost \& Goldstein
2021), as data infrastructures tend to be designed to answer questions
within national boundaries, and their use both draws upon and reinforces
territorial statecraft; as per Machen \& Nost (2021), this is
particularly true when the output of ``algorithmic thinking''
(\emph{e.g.} relying on machine learning to generate knowledge) can be
re-used for governance (\emph{e.g.} enacting conservation decisions at
the national scale). We therefore recognize that methods such as we
propose operate under the framework that contributed to the ongoing
biodiversity crisis (Adam 2014), reinforced environmental injustice
(Choudry 2013; Domínguez \& Luoma 2020), and on Turtle Island
especially, should be replaced by Indigenous principles of land
management (Eichhorn \emph{et al.} 2019; No'kmaq \emph{et al.} 2021). As
we see AI/ML being increasingly mobilized to generate knowledge that is
lacking for conservation decisions (\emph{e.g.} Lamba \emph{et al.}
2019; Mosebo Fernandes \emph{et al.} 2020), our discussion of these
tools need to go beyond the technical, and into the governance
consequences they can have.

\textbf{Acknowledgements:} We acknowledge that this study was conducted
on land within the traditional unceded territory of the Saint Lawrence
Iroquoian, Anishinabewaki, Mohawk, Huron-Wendat, and Omàmiwininiwak
nations. TP, TS, DC, and LP received funding from the Canadian Institue
for Ecology \& Evolution. FB is funded by the Institute for Data
Valorization (IVADO). TS, SB, and TP are funded by a donation from the
Courtois Foundation. CB was awarded a Mitacs Elevate Fellowship no.
IT12391, in partnership with fRI Research, and also acknowledges funding
from Alberta Innovates and the Forest Resources Improvement Association
of Alberta. M-JF acknowledges funding from NSERC Discovery Grant and
NSERC CRC. RR is funded by New Zealand's Biological Heritage Ngā Koiora
Tuku Iho National Science Challenge, administered by New Zealand
Ministry of Business, Innovation, and Employment. BM is funded by the
NSERC Alexander Graham Bell Canada Graduate Scholarship and the FRQNT
master's scholarship. LP acknowledges funding from NSERC Discovery Grant
(NSERC RGPIN-2019-05771). TP acknowledges financial support from NSERC
through the Discovery Grants and Discovery Accelerator Supplement
programs.

\hypertarget{refs}{}
\begin{CSLReferences}{1}{0}
\leavevmode\hypertarget{ref-Adam2014EleTre}{}%
Adam, R. (2014). \emph{Elephant treaties: The Colonial legacy of the
biodiversity crisis}. UPNE.

\leavevmode\hypertarget{ref-Albouy2019MarFis}{}%
Albouy, C., Archambault, P., Appeltans, W., Araújo, M.B., Beauchesne,
D., Cazelles, K., \emph{et al.} (2019). The marine fish food web is
globally connected. \emph{Nature Ecology \& Evolution}, 3, 1153--1161.

\leavevmode\hypertarget{ref-Buxton2021KeyInf}{}%
Buxton, R.T., Bennett, J.R., Reid, A.J., Shulman, C., Cooke, S.J.,
Francis, C.M., \emph{et al.} (2021). Key information needs to move from
knowledge to action for biodiversity conservation in Canada.
\emph{Biological Conservation}, 256, 108983.

\leavevmode\hypertarget{ref-Choudry2013SavBio}{}%
Choudry, A. (2013). Saving biodiversity, for whom and for what?
Conservation NGOs, complicity, colonialism and conquest in an era of
capitalist globalization. In: \emph{NGOization: Complicity,
contradictions and prospects}. Bloomsbury Publishing, pp. 24--44.

\leavevmode\hypertarget{ref-Csermely2004StrLin}{}%
Csermely, P. (2004). Strong links are important, but weak links
stabilize them. \emph{Trends in Biochemical Sciences}, 29, 331--334.

\leavevmode\hypertarget{ref-DallaRiva2016ExpEvo}{}%
Dalla Riva, G.V. \& Stouffer, D.B. (2016). Exploring the evolutionary
signature of food webs' backbones using functional traits. \emph{Oikos},
125, 446--456.

\leavevmode\hypertarget{ref-Dallas2017PreCry}{}%
Dallas, T., Park, A.W. \& Drake, J.M. (2017). Predicting cryptic links
in host-parasite networks. \emph{PLOS Computational Biology}, 13,
e1005557.

\leavevmode\hypertarget{ref-Dominguez2020DecCon}{}%
Domínguez, L. \& Luoma, C. (2020). Decolonising Conservation Policy: How
Colonial Land and Conservation Ideologies Persist and Perpetuate
Indigenous Injustices at the Expense of the Environment. \emph{Land}, 9,
65.

\leavevmode\hypertarget{ref-Dunne2006NetStr}{}%
Dunne, J.A. (2006). The Network Structure of Food Webs. In:
\emph{Ecological networks: Linking structure and dynamics} (eds. Dunne,
J.A. \& Pascual, M.). Oxford University Press, pp. 27--86.

\leavevmode\hypertarget{ref-Eichhorn2019SteDec}{}%
Eichhorn, M.P., Baker, K. \& Griffiths, M. (2019). Steps towards
decolonising biogeography. \emph{Frontiers of Biogeography}, 12, 1--7.

\leavevmode\hypertarget{ref-Fortin2021NetEco}{}%
Fortin, M.-J., Dale, M.R.T. \& Brimacombe, C. (2021). Network ecology in
dynamic landscapes. \emph{Proceedings of the Royal Society B: Biological
Sciences}, 288, rspb.2020.1889, 20201889.

\leavevmode\hypertarget{ref-Galiana2021SpaSca}{}%
Galiana, N., Barros, C., Braga, J., Ficetola, G.F., Maiorano, L.,
Thuiller, W., \emph{et al.} (2021). The spatial scaling of food web
structure across European biogeographical regions. \emph{Ecography},
n/a.

\leavevmode\hypertarget{ref-Galiana2019GeoVar}{}%
Galiana, N., Hawkins, B.A. \& Montoya, J.M. (2019). The geographical
variation of network structure is scale dependent: Understanding the
biotic specialization of hostparasitoid networks. \emph{Ecography}, 42,
1175--1187.

\leavevmode\hypertarget{ref-Galiana2018SpaSca}{}%
Galiana, N., Lurgi, M., Claramunt-López, B., Fortin, M.-J., Leroux, S.,
Cazelles, K., \emph{et al.} (2018). The spatial scaling of species
interaction networks. \emph{Nature Ecology \& Evolution}, 2, 782--790.

\leavevmode\hypertarget{ref-Gravel2018BriElt}{}%
Gravel, D., Baiser, B., Dunne, J.A., Kopelke, J.-P., Martinez, N.D.,
Nyman, T., \emph{et al.} (2018). Bringing Elton and Grinnell together: A
quantitative framework to represent the biogeography of ecological
interaction networks. \emph{Ecography}, 0.

\leavevmode\hypertarget{ref-Grunig2020CroFor}{}%
Grünig, M., Mazzi, D., Calanca, P., Karger, D.N. \& Pellissier, L.
(2020). Crop and forest pest metawebs shift towards increased linkage
and suitability overlap under climate change. \emph{Communications
Biology}, 3, 1--10.

\leavevmode\hypertarget{ref-Herbert1965Dun}{}%
Herbert, F. (1965). \emph{Dune}. 1st edn. Chilton Book Company,
Philadelphia.

\leavevmode\hypertarget{ref-Hoffmann2019MacLea}{}%
Hoffmann, J., Bar-Sinai, Y., Lee, L.M., Andrejevic, J., Mishra, S.,
Rubinstein, S.M., \emph{et al.} (2019). Machine learning in a
data-limited regime: Augmenting experiments with synthetic data uncovers
order in crumpled sheets. \emph{Science Advances}, 5, eaau6792.

\leavevmode\hypertarget{ref-Jordano2016SamNet}{}%
Jordano, P. (2016). Sampling networks of ecological interactions.
\emph{Functional Ecology}, 30, 1883--1893.

\leavevmode\hypertarget{ref-Lamba2019DeeLea}{}%
Lamba, A., Cassey, P., Segaran, R.R. \& Koh, L.P. (2019). Deep learning
for environmental conservation. \emph{Current Biology}, 29, R977--R982.

\leavevmode\hypertarget{ref-Machen2021ThiAlg}{}%
Machen, R. \& Nost, E. (2021). Thinking algorithmically: The making of
hegemonic knowledge in climate governance. \emph{Transactions of the
Institute of British Geographers}, 46, 555--569.

\leavevmode\hypertarget{ref-McLeod2021SamAsy}{}%
McLeod, A., Leroux, S.J., Gravel, D., Chu, C., Cirtwill, A.R., Fortin,
M.-J., \emph{et al.} (2021). Sampling and asymptotic network properties
of spatial multi-trophic networks. \emph{Oikos}, n/a.

\leavevmode\hypertarget{ref-Morales-Castilla2015InfBio}{}%
Morales-Castilla, I., Matias, M.G., Gravel, D. \& Araújo, M.B. (2015).
Inferring biotic interactions from proxies. \emph{Trends in Ecology \&
Evolution}, 30, 347--356.

\leavevmode\hypertarget{ref-MoseboFernandes2020MacLea}{}%
Mosebo Fernandes, A.C., Quintero Gonzalez, R., Lenihan-Clarke, M.A.,
Leslie Trotter, E.F. \& Jokar Arsanjani, J. (2020). Machine Learning for
Conservation Planning in a Changing Climate. \emph{Sustainability}, 12,
7657.

\leavevmode\hypertarget{ref-Neutel2002StaRea}{}%
Neutel, A.-M., Heesterbeek, J.A.P. \& de Ruiter, P.C. (2002). Stability
in Real Food Webs: Weak Links in Long Loops. \emph{Science}, 296,
1120--1123.

\leavevmode\hypertarget{ref-Nokmaq2021AwaSle}{}%
No'kmaq, M., Marshall, A., Beazley, K.F., Hum, J., joudry, shalan,
Papadopoulos, A., \emph{et al.} (2021). {``Awakening the sleeping
giant''}: Re-Indigenization principles for transforming biodiversity
conservation in Canada and beyond. \emph{FACETS}, 6, 839--869.

\leavevmode\hypertarget{ref-Nost2021PolEco}{}%
Nost, E. \& Goldstein, J.E. (2021). A political ecology of data.
\emph{Environment and Planning E: Nature and Space}, 25148486211043503.

\leavevmode\hypertarget{ref-OConnor2020UnvFoo}{}%
O'Connor, L.M.J., Pollock, L.J., Braga, J., Ficetola, G.F., Maiorano,
L., Martinez-Almoyna, C., \emph{et al.} (2020). Unveiling the food webs
of tetrapods across Europe through the prism of the Eltonian niche.
\emph{Journal of Biogeography}, 47, 181--192.

\leavevmode\hypertarget{ref-Pedersen2017SigCol}{}%
Pedersen, E.J., Thompson, P.L., Ball, R.A., Fortin, M.-J., Gouhier,
T.C., Link, H., \emph{et al.} (2017). Signatures of the collapse and
incipient recovery of an overexploited marine ecosystem. \emph{Royal
Society Open Science}, 4, 170215.

\leavevmode\hypertarget{ref-Poisot2016StrPro}{}%
Poisot, T., Cirtwill, A.R., Cazelles, K., Gravel, D., Fortin, M.-J. \&
Stouffer, D.B. (2016). The structure of probabilistic networks.
\emph{Methods in Ecology and Evolution}, 7, 303--312.

\leavevmode\hypertarget{ref-Poisot2015SpeWhy}{}%
Poisot, T., Stouffer, D.B. \& Gravel, D. (2015). Beyond species: Why
ecological interaction networks vary through space and time.
\emph{Oikos}, 124, 243--251.

\leavevmode\hypertarget{ref-Ray2021BioCri}{}%
Ray, J.C., Grimm, J. \& Olive, A. (2021). The biodiversity crisis in
Canada: Failures and challenges of federal and sub-national strategic
and legal frameworks. \emph{FACETS}, 6, 1044--1068.

\leavevmode\hypertarget{ref-Strydom2021RoaPre}{}%
Strydom, T., Catchen, M.D., Banville, F., Caron, D., Dansereau, G.,
Desjardins-Proulx, P., \emph{et al.} (2021). A roadmap towards
predicting species interaction networks (across space and time).
\emph{Philosophical Transactions of the Royal Society B: Biological
Sciences}, 376, 20210063.

\leavevmode\hypertarget{ref-Turak2017UsiEss}{}%
Turak, E., Brazill-Boast, J., Cooney, T., Drielsma, M., DelaCruz, J.,
Dunkerley, G., \emph{et al.} (2017). Using the essential biodiversity
variables framework to measure biodiversity change at national scale.
\emph{Biological Conservation}, SI:Measures of biodiversity, 213,
264--271.

\leavevmode\hypertarget{ref-Wood2015EffSpa}{}%
Wood, S.A., Russell, R., Hanson, D., Williams, R.J. \& Dunne, J.A.
(2015). Effects of spatial scale of sampling on food web structure.
\emph{Ecology and Evolution}, 5, 3769--3782.

\end{CSLReferences}

\end{document}
